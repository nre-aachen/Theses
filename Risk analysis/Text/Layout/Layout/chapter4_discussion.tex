	\chapter{Discussion}\label{cha:discussion}
	
	1D CASE:
	
				-abstract case: easy model construction and straightforward design of a loss function making basic assumptions and taking relative values that can simply be exchanged
			
				-so representative in a "relative" way, mostly appropriate to illustrate principles and benefits of the methodology
				
				-decision making/ estimation is defined by the design of the loss function which includes framework parameters which depend on the problem environment (e.g. market, technical constraints, etc.)
			
				-similar actors but different behaviors concerning risk: different decisions but same general loss function path (just different minima of EL)
			
				-additional information changes decision (BA) and EL
			
				-magnitude of change varies not only with the nature of the information (magnitude of uncertainty reduction) but also with the risk parameter
			
				-some actors might benefit more from some specific additional information than others
			
				-quantifiable value of information??
				
				
				
				
				- The GREATER the REDUCTION IN UNCERTAINTY, the LOWER the relevance of the risk-factor and difference in actor's preferences
				
				- accurate decision-making vs. gaining of information that yields encouraging results (vs "disappointing" results)