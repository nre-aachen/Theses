	\chapter{Discussion}\label{cha:discussion}
	Building upon the recent findings by \citet{delaVarga2016}, showing that structural modeling can be viewed as a problem of Bayesian inference, the aim of this work was to extend this approach by considering practical applications and the utility geological modeling might have in an economic context. A sector in which structural geological modeling is of central importance is hydrocarbon exploration. This field is characterized by the necessity to make decisions in the face of high risks and potentially high rewards. As these decisions are often closely linked to geological modeling and the estimation of reservoir-related values, this is work aimed to use this context to extend the Bayesian inference step in geological modeling by evaluating its results in terms of influence on respective decision-making. It was initially hypothesized that Bayesian updating via likelihoods and the resulting change of uncertainty in a model, should have significant effects on subsequent decision-making.\\
	To analyze this, geological models were interpreted as potential hydrocarbon systems. Algorithms for structural trap recognition and calculation of values relevant to respective decision-making were developed. Based on many iterations, probability distributions for such economic values (score and $ROV$, respectively) were generated. A custom loss function was designed to express the environment in which such values are to be estimated and to represent preferences of differently risk-affine decision-makers. These methods were applied for a conceptual 1D geological model and subsequently for a 3D structural model. Despite the great leap in complexity, the results from both models are widely similar.\\
	Scores in the 1D model and $ROV$ in the 3D model were chosen to represent a measure for the economic value realized in each model construction. As such, they were defined to depend on model parameters, either directly or indirectly by valuing their relations to other parameters or specific thresholds and conditions. Accordingly, their posterior probability distributions and respective uncertainties changed with the implementation of Bayesian inference.\\ 
	These changes related mostly to main modes in the distributions. For both model designs (1D and 3D), the prior value probability distributions were characterized by a distinct bimodality formed by widely separated modes of opposing values, i.e. one in the range of highly postive values, another at zero or negative values. By introducing likelihood functions, these modes were most often shifted or altered in their shape. The latter was observed as either an amplification of the mode, by raising its mean probability, while narrowing the range of probabilities (i.e. reducing the standard deviation), or the opposite, a diminishment of the mode up to total erasure of its probability. 
	Considering that these mode changes correlated with increased or decreased relevance of specific trap control mechanisms in the 3D model, it can be argued that each mode represents the occurrence of a specific mechanism-related scenario. These are mostly distinctly separated with little probability in between. This presumably is based on the assumption that traps are "filled-to-spill" in respect to spill and leak point, and that full leakage is assumed for any type of seal failure, resulting in an approximate discretization into respective scenarios.\\
	Bayesian inference primarily seems to lead to a change in uncertainty in respect to these scenarios or modes. The proportionality regarding these modes, their relative probabilities and distance to each other, in turn influence the overall uncertainty of the probability distribution as a whole. This was often indicated by the position of the mean and median relative to the modes. A mean located in the middle between two widely separated modes, which might be of low intrinsic uncertainty themselves, would indicate a much higher overall uncertainty, than a mean found in the middle of a narrow unimodal distribution.
	As the process of decision-making is based on these value distributions by applying the custom loss function respectively, it is proposed here, that this overall uncertainty might be referred to as "decision uncertainty".\\	
	The decision-making of differently risk-affine actors, i.e. the position and spread of Bayes actions, changed relative to the properties of value probability distributions and their inherent decision uncertainty. Distinct bimodality between two extremes, i.e. high decision uncertainty, resulted in a higher spread of Bayes estimators. Reduction of the distribution to one mode conversely led to the convergence of different Bayes actions. A decrease decision uncertainty furthermore was accompanied by a reduction in expected loss for each decision. Risk-friendly actors were the most robust in their decision-making in the face of possible trap failure. Eliminating this risk proved to be far less significant to the most risk-friendly actor. The respective estimate was generally found close to the positive mode and experienced only minor lateral shifting, after diminishing the probability of failure. Greater shifts were caused for this actor, in cases in which the positive mode itself was shifted laterally or erased completely. Conversely, the Bayes action of the most-averse individual was influenced greatly by the probability of the "worst scenario". Only a stark decrease of the respective mode encouraged this actor to bid on an estimate close to the risk-friendlier decision-makers. These observations match the principles and conditions conceived in the design of the loss function and the applied loss function. A clear difference in the decision-making relative to risk-affinity was observed.\\
	It can be argued that the degree of convergence of Bayes actions and their expected loss can be considered measures for the state of knowledge during the decision-making process. The better this is, the more similar the decisions of differently risk-affine actors are. It can be assumed that given perfect information, all actors would bid on the same estimate, the true value, and expect no loss. Thus, the relevance of the risk-factor decreases with higher reduction of decision uncertainty.\\			
	While in most cases the inclusion of additional information as likelihoods led to a reduction of decision uncertainty, this was not always the case. It appears that the change in decision uncertainty is not necessarily strictly aligned with the change in uncertainty regarding model parameters and their derivatives. In respect to this, it seems to be of central importance "where" in the model uncertainty is reduced, i.e. in which spatial area or regarding which model parameter. This appears to be of particular significance considering threshold values that lead to an abrupt cut-off between two extrema of decision options. In both types of model construction, 1D and 3D, this is directly related to sealing reliability. Thresholds regarding seal thickness (1D model) and Shale Smear Factor ($SSF_c$ in 3D model) were defined in a way that introduced a significant possibility of complete failure of a trap. Consequently, it was observed that reducing the uncertainty in a way that narrows the probability of a threshold-related parameter around its cut-off value, can lead to an amplification of the respective mode and thus emphasize the risk of complete failure. This can well be seen in the posterior $ROV$ distribution of model V (3D). Resulting Bayes actions were characterized by an increase in divergence and expected losses.\\
	Reducing model uncertainty in the "wrong" areas seems to simply lead to a transformation of this uncertainty in the realm of decision-making. Overall uncertainty seems to be conserved to a certain extent or even increased, as the duality of the decision problem is amplified in the form of a more stretched-out bimodal distribution and diverging Bayes actions. According to this, it should be of foremost importance for each actor, to reduce the uncertainty regarding threshold-related factors which might decide between "success" and complete failure of a project. Some type of information might improve the potential magnitude of a positive outcome, but maintain the risk of failure. For making better decisions, elimination of such high risks should be a priority. In the context of hydrocarbon traps, seal reliability is decisive.\\
	Such risks might be easily assessed, if they are dependent on only one or a few parameters, such as seal thickness in the 1D model. In other cases, they are derived from more complex parameter inter-relations, as is the case for the Shale Smear Factor, the ratio of displacement to seal thickness. Models IV and V (3D) showed that reduced uncertainty about the $SSF$ is not necessarily directly recognizable by information entropy visualization, especially if uncertainty regarding its parent parameters remains high. It follows that, to approach an effective mitigation of high risks, the complexities behind decisive factors need to be assessed and respective parent parameters and inter-dependencies need to be recognized. This might enable a better understanding of which type of information is missing and where in the model, additional data might be of use for improved decision-making.\\

	It has too be emphasized that the models constructed in this work were purely artificial and not based on real data. Nevertheless, the 3D model in particular was designed to include some typical structural characteristics related to hydrocarbon systems and algorithms were developed to consider the most common conditions that define structural traps. The fact that similar observations were made for the 1D, as well as the 3D model, indicates a certain degree of continuity with respect to representativity. However, it also has to be pointed out, that the uncertainties employed in the 3D model regarded positional values in the $z$-axis only, and were of one-dimensional nature. This might party be reason for the parallelism in the results. There was thereby also no real uncertainty concerning the overall structural shape, particularly anticlinal features and the position of the spill point in relation to the trap maximum.\\
	
	On algorithms
	
	For the purpose of a conceptual application, the customization of the loss functions was kept relatively simple. As mentioned above, despite its simplicity, the design of the function and the incorporation of a risk factor appears to have been suitable to achieve a clear distinction between differently risk-affine actors and respective behavior. In contrast to this, the use of standard loss function would have only returned mean or median as estimates. Customization of such loss functions offers a flexible approach to express different objective and subjective weighting factors that are specified to certain decision-making environments and decision-makers' perspectives. It is thereby possible to extend this design to attain very complex loss functions. As the loss function in this work was designed based on only a few basic assumptions, it might be more generally representative than a complex function which takes into account numerous very specific aspects. The consideration of more details, without a base of real data, would have furthermore required extensive speculation, which would have presumably impaired the generalization potential of respective observations.\\
	
	Considering the findings of this work, there is still many points that could be expanded on in future research. It would be of interest to apply the same overall concept and methods on an authentic case based on real datasets. Given a realistic economic scenario, including capital and operational expenditures of a project, possibly a full net-present-value ($NPV$) analysis could be conducted. Recoverable reserves could be replaced by the $NPV$ to evaluate modeling results and serve as a base for decision-making. A more elaborate respective loss function could be customized on the base of surveys, acquiring the specific preferences of one or more companies and thus attaining a better profile of the economic environment, as well as the individuals acting in it. Although the algorithms for automatic 3D hydrocarbon trap recognition were developed to fit the the model in this work, they presumably could be easily adapted to other structural cases or even engineered further to be universally applicable. 
	Furthermore, additional and different uncertainty parameters should be considered. A respective next step regarding this model would be the incorporation of uncertainties which to a wider extent affect structural shapes in all three dimension. Otherwise, non-structural reservoir parameters could be included as uncertain, in particular those which are part of the $ROV$ equation. 
	Hydrocarbon systems and petroleum exploration as a sector were chosen for an exemplary application in this work, but other settings can be found, in which geological modeling is of central significance for decision-making. Related to the context and results of this work would be subsurface storage of fluids in reservoir, such as Carbon Capture and Storage. Questions of storage capacity and safety deal with similar conditions and geological problems, as the ones presented in this here, most importantly seal reliability and the risk of leakage. In respect to this, the models and approaches in this work might provide a base to raise new cases that can be analyzed.