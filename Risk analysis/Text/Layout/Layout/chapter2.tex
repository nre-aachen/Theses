    \chapter{First Real Chapter}

    This is a demonstration chapter. I will explain some of the possibilities of \LaTeX. Here something will be shown of control theory, 'the transfer function' \lsymb{$H(s)$}{Transfer function}. Subscripts and superscripts can be put in the nomenclature \index{nomenclature} list. \supers{max}{Maximum} \subs{min}{Minimum} Other things can also be added to the nomenclature list, like explanations of symbols being used throughout the thesis. \others{[kts]}{Knots} \others{$^{\circ}$, [deg]}{Degrees}

        \section{First section}

        This is the section. Referring to equations, figures and tables can easily be done by the commands \verb"\eqnref{}",
        \verb"\figref{}" and \verb"\tabref{}".
        \begin{equation}\label{eq:First}
        H(s) = \frac{1}{s+2}
        \end{equation}

        You see? Refer to equations like this \eqnref{eq:First}, i.e. the name of the label you have given the specific equation, figure or table.
        
        \subsection{The first subsection}
  
        Now I demonstrate, numbering equations, using subequations:
	  \begin{subequations}
		\begin{eqnarray}
    \label{2eq1d1}
	  \nabla\times\mathbf{L}  &=& \frac{\partial\mathbf{G}}{\partial t} \\
    \label{2eq1d2}
  	\nabla\times\mathbf{G}  &=& \frac{\partial\mathbf{L}}{\partial t} + \mathbf{J} \\
    \label{2eq1d3}
    \mathbf{G}              &=& \sigma\mathbf{J}
		\end{eqnarray}
			  \end{subequations}
	
				Or we can make matrices:
				\begin{equation}
				\mathbf{Q}_{12}=\left[\begin{array}{ccc}
		          0  &     1          &  0 \\
	            1  &     0          &  1 \\
	            0  &     1          &  0 \\
       	\end{array}\right]\quad
        \nonumber
		    \label{2eq1cf}
		    \end{equation}
		    This can also be done using the \verb"\align{}" command. Equation arrays are also possible:
     		\begin{eqnarray}
    		\label{2eq1e1}
	  \nabla\times\mathbf{L}  &=& \frac{\partial\mathbf{G}}{\partial t} \\
	    	\label{2eq1e2}
  	\nabla\times\mathbf{G}  &=& \frac{\partial\mathbf{L}}{\partial t} + \mathbf{J} \\
		    \label{2eq1e3}
    \mathbf{G}              &=& \sigma\mathbf{J}
		  \end{eqnarray}

       
        \subsubsection[Subsection Short Title]{The first sub-subsection with a very very very long title, but in the table of contents one can only see the short title in square brackets}

                Impressed by the capabilities? \index{Nicecapabilities}
                If you want to know more about the capabilities of \LaTeX, take a look at the "\textbf{The Not So Short Introduction to \LaTeXe}", which can be found on the internet.

    \paragraph{Next paragraph.}
    \begin{figure}[htbp]  % fig 6.3 page 186 Turbulence and Remous
    \centering
    \includegraphics[width=11cm]{vertstab.eps}
    \caption{Stability conditions for the vertical stability of saturated and unsaturated air.} %this is how it shows up in the List of Figures
    \label{f:verticalstab}
    \end{figure}
.\\    
And finally I end this example file with a table which will be centered in the middle of the following page
\begin{table}[hp]
\centering
\renewcommand{\arraystretch}{1.5}
\begin{tabular}{|ll|}
\hline
\multicolumn{2}{|c|}{\bf\sffamily {\it Data} files listed in a table }\\
\hline\hline
\multicolumn{2}{|l|}{\bf\sffamily First part} \\
Fabracadabra.m	& - Saturation computation\\
Fobracadabra.m	& - Pressure computation\\
Fibricadibri.m	& - Permeability computation\\
&\\
\multicolumn{2}{|l|}{\bf\sffamily Structural rock model} \\
struct.m	& - Rock structural data using symmetric boundary condition \\
bstruct.m	& - Rock structural data using anti-symmetric boundary condition \\
\hline
\end{tabular}
\caption{Good-looking program data deck files.} %this is how it shows up in the List of Tables
\label{tbl:tbl}
\end{table}
